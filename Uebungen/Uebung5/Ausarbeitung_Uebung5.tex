\documentclass[a4paper,11pt,titlepage]{article}

\usepackage{ucs}
\usepackage[german,ngerman]{babel}
\usepackage{fontenc}
\usepackage[pdftex]{graphicx}
\usepackage[pdftex]{hyperref}
\usepackage{listings}
\usepackage{xcolor}
\definecolor{codegreen}{rgb}{0,0.6,0}


\lstdefinestyle{mystyle}{  
    commentstyle=\color{codegreen},
    keywordstyle=\color{blue},
    basicstyle=\ttfamily\footnotesize,
    breakatwhitespace=false,         
    breaklines=true,                 
    captionpos=b,                    
    keepspaces=true,                 
    numbers=left,                    
    numbersep=5pt,                  
    showspaces=false,                
    showstringspaces=false,
    showtabs=false,                  
    tabsize=2
}

\lstset{style=mystyle}

\begin{document}

% hier aktuelle Uebungsnummer einfuegen
\title{Einf\"uhrung in die Informatik\\
Ausarbeitung \"Ubung 5}

% Namen der Bearbeiter einfuegen

\author{Jakob Schulz}

% aktuelles Datum einfuegen

\date{\today}

\maketitle{\thispagestyle{plain}}

\section{Zufallszahlen 1}
Ausarbeitung:
\lstinputlisting[language=c++]{task1.cpp}\newpage
\section{Zufallszahlen 2}
Ausarbeitung:
\lstinputlisting[language=c++]{task2.cpp}\newpage
\section{Bin"ardarstellung}
Ausarbeitung:
\lstinputlisting[language=c++]{task3.cpp}\newpage
\section{L"ange einer Zeichenkette}
Ausarbeitung:
\lstinputlisting[language=c++]{task4.cpp}\newpage
\section{Zeichenkettenvergleich}
Ausarbeitung:
\lstinputlisting[language=c++]{task5.cpp}\newpage
\section{Zeichenkette umdrehen}
Ausarbeitung:
\lstinputlisting[language=c++]{task6.cpp}\newpage
\section{Zeichenketten ohne Vokale}
Ausarbeitung:
\lstinputlisting[language=c++]{task7.cpp}\newpage
\section{Zeichenketten zusammenf"ugen}
Ausarbeitung:
\lstinputlisting[language=c++]{task8.cpp}\newpage

\end{document}
