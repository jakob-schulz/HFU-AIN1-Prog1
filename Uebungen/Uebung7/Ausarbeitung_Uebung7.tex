\documentclass[a4paper,11pt,titlepage]{article}

\usepackage{ucs}
\usepackage[german,ngerman]{babel}
\usepackage{fontenc}
\usepackage[pdftex]{graphicx}
\usepackage[pdftex]{hyperref}
\usepackage{color}
\usepackage{xcolor}
\usepackage{listings}


\definecolor{dunkelblau}{RGB}{16, 55, 188}
\definecolor{orange}{RGB}{255, 60, 0}
\definecolor{gruen}{RGB}{18, 118, 34}
\definecolor{gelb}{RGB}{255, 200, 0}
\definecolor{lila}{RGB}{147, 18, 114}

\lstdefinestyle{c++}{
language=C++,
commentstyle=\color{gruen}, 
keywordstyle =\color{dunkelblau}, 
stringstyle=\color{orange},
literate=
    {\{}{{\textcolor{gelb}{\{}}}1
    {\}}{{\textcolor{gelb}{\}}}}1
    {[}{{\textcolor{lila}{[}}}1
    {]}{{\textcolor{lila}{]}}}1,
%Bis hier hin Farbgebung
frame=single, % Umrandung des Codes
rulecolor=\color{lightgray},
numbers=left, % Nummerierung hinzufügen (links)
numberstyle=\tiny, % Stil der Zeilennummern
stepnumber=1, % Schrittzahl für die Nummerierung
numbersep=5pt, % Abstand zwischen Nummerierung und Code
basicstyle=\sffamily, % Ändert die Schriftart des Codes
tabsize = 4, %Tab-Abstand
breaklines=true, %Zeilenumbruch
showstringspaces=false
}


\begin{document}

% hier aktuelle Uebungsnummer einfuegen
\title{Programmieren "Ubung 7}

% Namen der Bearbeiter einfuegen

\author{Jakob Schulz}

% aktuelles Datum einfuegen

\date{\today}

\maketitle{\thispagestyle{plain}}

\section{Taschenrechnerfunktion}
Ausarbeitung:
\lstinputlisting[style = c++]{task1.cpp}\newpage
\section{Kommandozeilenrechner}
Ausarbeitung:
\lstinputlisting[style = c++]{task2.cpp}\newpage
\section{Gr"oßter gemeinsamer Teiler}
Ausarbeitung:
\lstinputlisting[style = c++]{task3.cpp}\newpage
\section{Primzahlen}
Ausarbeitung:
%\lstinputlisting[style = c++]{task4.cpp}\newpage
\begin{lstlisting}[style = c++]
#include <stdio.h>

// Prototypen
bool prim(int x);

int main()
{
    int input;
    printf("Zahl, die ueberprueft werden soll:");
    scanf("%i", &input);
    if (prim(input))
    {
        printf("Die Zahl ist prim.");
    }
    else
    {
        printf("Die Zahl ist nicht prim.");
    }
    return 0;
}

// ueberprueft, ob eine eingegebene Zahl eine Primzahl ist
bool prim(int x)
{
    for (int quotient = 2; quotient * quotient <= x; quotient++)
    {
        if ((x % quotient) == 0)
        {
            return false;
        }
    }
    return true;
}
\end{lstlisting}
Frage: Bis wohin muss man Teiler "uperpr"ufen?\\
Die h"ochste Zahl, die man Pr"ufen muss, ob sie Teiler der eingegeben Zahl ist, ist die Wurzel der eingegebenen Zahl\\
Grund: Zieht man die Wurzel sind beide Teiler gr"oßtm"oglich. F"ur jeden Teiler, der gr"oßer als die Wurzel ist, ist der andere kleiner als die Wurzel.\\
Weiteres Beispiel:\\
F"ur eine eingegebene Zahl t m"usste gelten:\\
$\sqrt{t+x}\cdot \sqrt{t} = t$\\
Ergebnis der Gleichung:\\
$t\cdot x = t.$\\
Somit gibt es keine Teilerkombination bestehend aus einer Zahl gr"oßer Wurzel n und gleich Wurzel n.
\section{Worte und Zeichen 1}
Ausarbeitung:
\lstinputlisting[style = c++]{task5.cpp}\newpage
\section{Worte und Zeichen 2}
Ausarbeitung:
\lstinputlisting[style = c++]{task6.cpp}\newpage


\end{document}
