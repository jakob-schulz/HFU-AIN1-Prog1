\documentclass[a4paper,11pt,titlepage]{article}

\usepackage{ucs}
\usepackage[german,ngerman]{babel}
\usepackage{fontenc}
\usepackage[pdftex]{graphicx}
\usepackage[pdftex]{hyperref}
\usepackage{color}
\usepackage{xcolor}
\usepackage{listings}


\definecolor{dunkelblau}{RGB}{16, 55, 188}
\definecolor{orange}{RGB}{255, 60, 0}
\definecolor{gruen}{RGB}{18, 118, 34}
\definecolor{gelb}{RGB}{255, 200, 0}
\definecolor{lila}{RGB}{147, 18, 114}

\lstdefinestyle{c++}{
language=C++,
commentstyle=\color{gruen}, 
keywordstyle =\color{dunkelblau}, 
stringstyle=\color{orange},
literate=
    {\{}{{\textcolor{gelb}{\{}}}1
    {\}}{{\textcolor{gelb}{\}}}}1
    {[}{{\textcolor{lila}{[}}}1
    {]}{{\textcolor{lila}{]}}}1,
%Bis hier hin Farbgebung
frame=single, % Umrandung des Codes
rulecolor=\color{lightgray},
numbers=left, % Nummerierung hinzufügen (links)
numberstyle=\tiny, % Stil der Zeilennummern
stepnumber=1, % Schrittzahl für die Nummerierung
numbersep=5pt, % Abstand zwischen Nummerierung und Code
basicstyle=\sffamily, % Ändert die Schriftart des Codes
tabsize = 4, %Tab-Abstand
breaklines=true, %Zeilenumbruch
showstringspaces=false
}


\definecolor{hellblau}{RGB}{100,149,237}
\begin{document}

% hier aktuelle Uebungsnummer einfuegen
\title{Programmieren "Ubung 8}

% Namen der Bearbeiter einfuegen

\author{Jakob Schulz}

% aktuelles Datum einfuegen

\date{\today}

\maketitle{\thispagestyle{plain}}

\section{Strukturen}
Ausarbeitung:
\lstinputlisting[style = c++]{task1.c}\newpage
\section{Arrays und Strukturen}
Ausarbeitung:
\lstinputlisting[style = c++]{task2.c}\newpage
\section{Datum}
Ausarbeitung:
\lstinputlisting[style = c++]{task3.cpp}\newpage
\section{Funktionen mit Strukturen als Parameter}
Ausarbeitung:
\lstinputlisting[style = c++]{task4.cpp}\newpage
\section{Vektorfunktionen}
Ausarbeitung:
\lstinputlisting[style = c++]{task5.cpp}\newpage
\section{Personendatenbank}
Ausarbeitung:
\lstinputlisting[style = c++]{task6.cpp}\newpage
\section{Geschachtelte Schleifen}
Array Zahlen zu Beginn: (2)(1)(0)(2)(1)\\
\begin{tabular}{lll}
Anzahl besuchte innere for-Schleife&Zusammensetzung der Zahlen&Array\\
Erstes Mal (i==0):& (\textcolor{hellblau}{2+1})(1)(\textcolor{hellblau}{3+0})(2)(\textcolor{hellblau}{3+3})&= (3)(1)(3)(2)(6)\\
Zweites Mal (i==1):& (\textcolor{hellblau}{3+6})(1)(\textcolor{hellblau}{9+3})(2)(\textcolor{hellblau}{9+9})&= (9)(1)(12)(2)(18)\\
Drittes Mal (i==2):&(\textcolor{hellblau}{1+18})(1)(\textcolor{hellblau}{1+12})(2)(\textcolor{hellblau}{1+19})&= (19)(1)(13)(2)(20)\\
Viertes Mal (i==3):&(\textcolor{hellblau}{1+20})(1)(\textcolor{hellblau}{1+13})(2)(\textcolor{hellblau}{1+21})&= (21)(1)(14)(2)(22)\\
F"unftes Mal (i==4):&(\textcolor{hellblau}{14+22})(1)(\textcolor{hellblau}{14+14})(2)(\textcolor{hellblau}{28+36})&= (36)(1)(28)(2)(64)
\end{tabular}
\\
\\
Die blauen Elemente geben an, welche Spalten ge"andert werden\\
\lstinputlisting[style = c++]{task7.cpp}
\end{document}
