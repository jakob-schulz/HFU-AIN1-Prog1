\documentclass[a4paper,11pt,titlepage]{article}

\usepackage{ucs}
\usepackage[german,ngerman]{babel}
\usepackage{fontenc}
\usepackage[pdftex]{graphicx}
\usepackage[pdftex]{hyperref}
\usepackage{listings}
\usepackage{xcolor}
\definecolor{codegreen}{rgb}{0,0.6,0}

\lstdefinestyle{mystyle}{  
    commentstyle=\color{codegreen},
    keywordstyle=\color{blue},
    basicstyle=\ttfamily\footnotesize,
    breakatwhitespace=false,         
    breaklines=true,                 
    captionpos=b,                    
    keepspaces=true,                 
    numbers=left,                    
    numbersep=5pt,                  
    showspaces=false,                
    showstringspaces=false,
    showtabs=false,                  
    tabsize = 2
}
\lstset{style=mystyle}

\begin{document}

% hier aktuelle Uebungsnummer einfuegen
\title{Einf\"uhrung in die Informatik\\
Ausarbeitung \"Ubung 2}

% Namen der Bearbeiter einfuegen

\author{Jakob Schulz}

% aktuelles Datum einfuegen

\date{\today}

\maketitle{\thispagestyle{plain}}

\section{Ausdr"ucke}
\begin{tabular}{lll}
int n;\\
n = 17 - 2 * 7 + 9 \% 6 ;&n = 6& Y\\
n = ( 17 - 2 ) * ( 7 + 9 ) \% 6;&n = 0& Y\\
n = ( 17 - 2 ) * ( ( 7 + 9 ) \% 6 ) ;& n = 60 &Y\\
n = ( ( ( ( ( ( 1 7 - 2 ) * 7 ) - 9 ) * 7 ) + 9 ) \% 6 ) ; &n = 3 &Y\\
n = 17 - ( 2 * ( 7 + ( 9 \% 6 ) ) ) ;&n = -3 &Y\\
n = 17 / ( 5 / 3 ) * 4 ;	 &n = 68 &Y\\
n = ( 17 / 5 ) * ( 5 / 17 ) ; &n = 3 &F $\Rightarrow$ n = 0\\
\\
f l o a t f ;\\
f = 1 7.0 / ( 5 / 3 ) * 4;& f = 68& Y\\
f = ( 17.0 / 5 ) * ( 5 / 1 7 );&f = 0& Y\\
f = 1 * ( 1.0 / 3 ) * 3;& f = 1&Y\\
f = 1.5 e2 * 1.5 e2;&f = 2.25e4&Y\\
f = 1.5 e2 * 1.5 e-2;&f = 2.25e0&Y\\
f = 1.5 e- 2 * 1.5 e-2;&2.25e-2&(F)$\Rightarrow$ 0.000224999996\\
\\
char c;\\
c = 'a' + 5;&	c = 102&Y\\
c = '0' + 9;&c =  9&F $\Rightarrow$ 57\\
c = '0' + 9 / 2;&c = 4&F$\Rightarrow$52\\
c = '0' + 9 - 2;&c = 7&F$\Rightarrow$55\\
c = '0' + '9' ;	&c = 9&F$\Rightarrow$ 105\\
\\
bool b;\\
c = '5';\\							
b = ( $c >= `0`\&\& c <= `9`$ );&b = true&Y\\
b = ( $c >= 0 \&\& c <= 9$) ;&b = false&Y\\	
b = ( $c >= `0` || c <= `9`$);&b = true&Y\\
b = ( $c >= 0 | | c <= 9$ );&b = false&Y\\
Zusatz: ' sehen eigentlich ander aus\\
\\
int m = 44;\\
m = 44 $>>$ 2;&m = 10&(F) $\Rightarrow$ 11\\
m = 44 $<<$ 1;&m = 88&Y\\
m = 1 $<<$ 10;&m = 1.024e3&Y\\
m = 1 $<<$ 32;&m = -4.294967296e9&F$\Rightarrow$0
\end{tabular}
\section{Arithmetik von ganzen Zahlen}
Bei sehr großen Zahlen entstehen Fehler$\Rightarrow$ Grund dafür ist, dass ein Overflow entsteht
\section{Konvertierungen}
\begin{itemize}
\item 33
\item 33
\item 33
\item 0
\end{itemize}
\section{Shift-Operatoren}
\begin{lstlisting}[language=c++]
int main()
{
    int n = 12;
    n = 12 <<3;
    int i = 10000;
    i = 10000 >> 2;
    int k = 5;
    k = 300 << 5;
}	
\end{lstlisting}
n = 96, i = 2500, k = 9600
\section{Speicherplatzgr"oßen}
\lstinputlisting[language=c++]{task5.cpp}
\end{document}
