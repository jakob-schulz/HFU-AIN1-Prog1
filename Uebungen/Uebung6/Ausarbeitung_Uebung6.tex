\documentclass[a4paper,11pt,titlepage]{article}

\usepackage{ucs}
\usepackage[german,ngerman]{babel}
\usepackage{fontenc}
\usepackage[pdftex]{graphicx}
\usepackage[pdftex]{hyperref}
\usepackage{color}
\usepackage{xcolor}
\usepackage{listings}


\definecolor{dunkelblau}{RGB}{16, 55, 188}
\definecolor{orange}{RGB}{255, 60, 0}
\definecolor{gruen}{RGB}{18, 118, 34}
\definecolor{gelb}{RGB}{255, 200, 0}
\definecolor{lila}{RGB}{147, 18, 114}

\lstdefinestyle{c++}{
language=C++,
commentstyle=\color{gruen}, 
keywordstyle =\color{dunkelblau}, 
stringstyle=\color{orange},
literate=
    {\{}{{\textcolor{gelb}{\{}}}1
    {\}}{{\textcolor{gelb}{\}}}}1
    {[}{{\textcolor{lila}{[}}}1
    {]}{{\textcolor{lila}{]}}}1,
%Bis hier hin Farbgebung
frame=single, % Umrandung des Codes
rulecolor=\color{lightgray},
numbers=left, % Nummerierung hinzufügen (links)
numberstyle=\tiny, % Stil der Zeilennummern
stepnumber=1, % Schrittzahl für die Nummerierung
numbersep=5pt, % Abstand zwischen Nummerierung und Code
basicstyle=\sffamily, % Ändert die Schriftart des Codes
tabsize = 4, %Tab-Abstand
breaklines=true, %Zeilenumbruch
showstringspaces=false
}


\begin{document}

% hier aktuelle Uebungsnummer einfuegen
\title{Programmieren "Ubung 6}

% Namen der Bearbeiter einfuegen

\author{Jakob Schulz}

% aktuelles Datum einfuegen

\date{\today}

\maketitle{\thispagestyle{plain}}

\section{Tauschen von Werten}
Ausarbeitung:
\lstinputlisting[style = c++]{task1.cpp}\newpage
\section{Arrays ohne [] -die Erste}
Ausarbeitung:
\lstinputlisting[style = c++]{task2.cpp}\newpage
\section{Arrays ohne [] -die Zweite}
Ausarbeitung:
\lstinputlisting[style = c++]{task3.cpp}\newpage
\section{Quadratfunktion}
Ausarbeitung:
\lstinputlisting[style = c++]{task4.cpp}\newpage
\section{String umdrehen}
Ausarbeitung:
\lstinputlisting[style = c++]{task5.cpp}\newpage
\section{Zweierpotenzen}
Ausarbeitung:
\lstinputlisting[style = c++]{task6.cpp}\newpage
\section{Teilstrings z"ahlen}
Ausarbeitung:
\lstinputlisting[style = c++]{task7.cpp}\newpage

\end{document}
