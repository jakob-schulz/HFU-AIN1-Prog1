\documentclass[a4paper,11pt,titlepage]{article}

\usepackage{ucs}
\usepackage[german,ngerman]{babel}
\usepackage{fontenc}
\usepackage[pdftex]{graphicx}
\usepackage[pdftex]{hyperref}
\usepackage{listings}
\usepackage{xcolor}
\definecolor{codegreen}{rgb}{0,0.6,0}


\lstdefinestyle{mystyle}{  
    commentstyle=\color{codegreen},
    keywordstyle=\color{blue},
    basicstyle=\ttfamily\footnotesize,
    breakatwhitespace=false,         
    breaklines=true,                 
    captionpos=b,                    
    keepspaces=true,                 
    numbers=left,                    
    numbersep=5pt,                  
    showspaces=false,                
    showstringspaces=false,
    showtabs=false,                  
    tabsize=2
}

\lstset{style=mystyle}

\begin{document}

% hier aktuelle Uebungsnummer einfuegen
\title{Einf\"uhrung in die Informatik\\
Ausarbeitung \"Ubung 2}

% Namen der Bearbeiter einfuegen

\author{hier steht dann der oder die Namen(n)}

% aktuelles Datum einfuegen

\date{\today}

\maketitle{\thispagestyle{plain}}

\section{"Ubung}
\begin{verbatim}
for (int zaehler = 0; zaehler != 10; zaehler = zaehler + 1)
\end{verbatim}
Die Schleife wird 10 mal durchlaufen, wobei die Z"ahlervariable alle Werte von einschließlich 0 bis einschließlich 10 annimt. Wenn Z"ahlervariable jedoch 10 ist, wird Schleife nicht mehr durchlaufen.
\begin{verbatim}

for (int n = 10; n > 0; n = n - 1)
\end{verbatim}
Die Schleife wird 10 mal durchlaufen, wobei die Z"ahlervariable alle Werte von einschließlich 0 bis einschließlich 10 annimt. Wenn Z"ahlervariable jedoch 0 ist, wird Schleife nicht mehr durchlaufen.
\begin{verbatim}

for (int x = 1; x <= 15; x = x + 3)
\end{verbatim}
Die Schleife wird 5 mal durchlaufen, wobei die Z"ahlervariable die Werte von einschließlich 1, 4, 7, 10 und 13 annimt.
\begin{verbatim}

for (int x = 1; x != 15; x = x + 3)
\end{verbatim}
Es wurde eine Endlossschleife erzeugt, denn die Z"ahlervariable ist immmer ungleich 15. Die Zählervariable nimmt folgende Werte an: 1, 4, 7, 10, 13, 16, \dots
\begin{verbatim}

for (int x = 1; x == 15; x = x + 3)
\end{verbatim}
Die Schleife wird nie durchlaufen, denn die Z"ahlervariable ist bei der Initialisierung ungleich 15, sodass die Bedingung der Schleife nicht erfüllt ist und die Schleife kein einziges Mal ausgef"uhrt wird.
\begin{verbatim}

for (char c = '5'; c <= '9'; c = c + 2)
\end{verbatim}
Die Schleife wird 3 Mal durchlaufen, wobei die Z"ahlervariable die Werte 5, 7 und 9 annimt.\newpage
\section{"Ubung}
Ausarbeitung:
\lstinputlisting[language=c++]{task2.cpp}\newpage
\section{"Ubung}
Ausarbeitung:
\lstinputlisting[language=c++]{task3.cpp}\newpage
\section{"Ubung}
Ausarbeitung:
\lstinputlisting[language=c++]{task4.cpp}\newpage
\section{"Ubung}
Ausarbeitung:
\lstinputlisting[language=c++]{task5.cpp}\newpage
\section{"Ubung}
Ausarbeitung:
\lstinputlisting[language=c++]{task6.cpp}

\end{document}
